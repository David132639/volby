\documentclass[12pt,a4paper]{article}
\usepackage[utf8]{inputenc}
\usepackage[czech]{babel}
\usepackage[T1]{fontenc}
\usepackage{amsmath}
\usepackage{amsfonts}
\usepackage{amssymb}
\usepackage{graphicx}
\usepackage{lmodern}
\usepackage[left=2.5cm,right=2cm,top=2.5cm,bottom=2cm]{geometry}
\author{David Zacek}
\title{Volby a matematika}
\begin{document}
Cilem teto prace je analyzovat soucasny volebni system v CR a navrhnout jeho upravy, ktere by ho udelali spravedlivejsim. 

Volebnim systemem budu v teto praci nazyvat postup, dle ktereho je mozne mezi strany ucastnici se voleb rozdelit na zaklade poctu hlasu ktere obdrzely nedelitelne mandaty ve volenem sboru. Tyto systemu mohou ale byt analogicky pouzity i v dalsich situacich, nebot zobecnene jde o rozdeleni k nedelitelnych objektu mezi n subjeku, s tim ze kazdy subjekt ma ciselne ohodnoceni a rozdeleni by melo byt provedeno tak, aby pomer obdrzenych objeku a vlastniho ohodnoceni kazdeho subjektu byl co nejpodobnejsi.  Historie volebnich systemu saha do konce 18. stoleti \cite, kdy vznikaly spojene staty americke. V te dobe neslo o deleni mandatu mezi strany dle volebniho zisku, ale o deleni mandatu mezi 
\end{document}
