\documentclass[12pt,a4paper]{report}
\usepackage[utf8]{inputenc}
\usepackage[czech]{babel}
\usepackage[T1]{fontenc}
\usepackage{amsmath}
\usepackage{hyperref}
\usepackage{amsfonts}
\usepackage{amssymb}
\usepackage{graphicx}
\usepackage{lmodern}
\usepackage{multirow}
\usepackage{booktabs}
\usepackage[backend=biber, style=numeric, citestyle=numeric]{biblatex}
\addbibresource{citace.bib}
\usepackage{indentfirst}
\usepackage{setspace}
\onehalfspacing
\usepackage[left=2.5cm,right=2cm,top=2.5cm,bottom=2cm]{geometry}
\makeindex
\begin{document}
\begin{titlepage}
	\begin{center}
	{\large Gymnázium Christiana Dopplera, Zborovská 45, Praha 5 \par}
	\vfill
	\par\vspace{1cm}
	{\scshape\LARGE ROČNÍKOVÁ PRÁCE \par}
	{\huge\bfseries Analyza systemu voleb do Poslanecke snemovny Ceske republiky\par}
	\vfill
\end{center}
Vypracoval: David Žáček \\
Třída: 8. M \\
Školní rok: 2016/2017 \\
Seminář : Matematický seminář \\
	\vfill
\end{titlepage}

\vspace*{\fill}
\begin{quote}
Prohlašuji, že jsem svou ročníkovou práci napsal samostatně a výhradně s
použitím 
citovaných pramenů. Souhlasím s
využíváním 
práce na Gymnáziu Christiana Dopplera 
pro studijní účely. \\
\end{quote}

V Praze dne \today \hfill David Žáček

\vspace*{\fill}
\thispagestyle{empty}
\newpage
\tableofcontents
\newpage
\chapter{Uvod}
\section{Cile}
Cilem teto prace je analyzovat soucasny volebni system  pro volby do PSP CR a navrhnout jeho upravy, ktere by ho udelali spravedlivejsim. 
Volebnim systemem budu v teto praci nazyvat postup, dle ktereho se mezi strany ucastnici se voleb rozdeluji mandaty na zaklade poctu hlasu ktere obdrzely. 
Nebude zde zkoumana umela hranice pro zahrnuti strany do rozdelovani mandatu, u nas znama jako \uv{petiprocentni hranice}.
\footnote{Petiprocentni hranice ve skutecnosti neni zcela presny nazev. Hranice je jednak vyssi pro koalice stran, atake muze byt snizena pokud by ji neprekrocili alespon dve strany.\autocite{ZAK}}
Tato hranice existuje ve vetsine statu ktere vyuzivaji pomerovy system a byla predmetem mnoha zkoumani a teto prace se bude venovat jinym otazkam.
Hlasy pro strany ktere hranice nedosahly budou v praci povazovany za neodevzdane ci neplatne.
I kdyz budou zkoumany volby v CR, mohou zavery byt aplikovany i v jinych statech. 
\section{Deleni volebnich systemu}
Volebni systemy lze delit na vetsinove a pomerove. Vetsinove systemy jsou ty, ktere se ptaji pouze na otazku, kdo ma nejvic a jsou nezavisle na poctech hlasu dalsich kandidatu.
Vitezi pouze ten s nejvetsim poctem hlasu.
Pro volbu vetsiho poctu osob se zpravila voli kazda disjunktni skupina obyvatel sveho jednoho zastupce.
\footnote{Existuje i moznost aby zvoleno bylo v oblasti vice kandidatu s njvice hlasy, to se deje napriklad pri volbach ve state Tuvalu.\autocite{TUV}}
Prikladem vetsinoveho systemu jsou parlamentni volby v UK a USA, kde kazdou oblast reprezentuje ten kandidat, ktery v ni ziskal nejvice hlasu.
Tento postup ma ale pomerne zasadni vadu, zcela ignoruje velke mnozstvi hlasu.
To lze demonstrovat na hypotetickem na prikladu. Ve volbach se o prizen volicu uchazi pet nazoru na fungovani zeme, ktere budu oznacovat A az E.
Celkem se voli 8 poslancu.
Kazdy nazor ma sve poporovatele, nazor A podporuje 25\% obyvatel, nazor B 21\%, C 19\%, D 18\% a E 17\%.
Predpokladejme, ze vsechny nazory jsou u vsech skupinach obyvatel ve stejnem pomeru, a predpokladame dokonalou spolupraci vsech priznivcu kazdeho nazoru a neochotu pracovat se stoupenci jineho nazoru.
Pokud bysme vytvorili 8 skupin obyvatel (napriklad podle bydliste) a kazdou nechali rozhodnout o jednom poslanci, vsech 8 poslancu by nakonec podporovalo nazor A.
To proto, ze A by v kazdem hlasovani navrhlo jedineho kandidata, ktery by s 25\% podporou zisal vic hlasu nez libovolny kandidat dalsi strany. 
Tento vysledek se ale nezda velmi spravedlivy z toho pohledu, ze proti nazoru A hlasovaly tri ctvrtiny obyvatel a presto ziskali jeho stoupenci vsechna kresla v parlamentu.

Pomerove zastoupeni tento problem resi.
Namisto otazky kdo ziskal nejvice se pta, kdo ziskal jakou cast. Kandidati se stejnym nazorem pak do voleb vstupuji jako kandidujici politicka strana, volici hlasuji pro strany a kazda strana pak dostane tolik zaptupitelu, aby jeji procentualni zisk ve volbach co nejvice odpovidal zisku zastupitelu.
Otazka jak provest rozdeleni mandatu dle vysledku voleb bude hlavnim predmetem zkoumani teto prace.
Nyni muzeme pro predstavu udelit poslance stranam (odpovidajicim nazorum) A az E napr tak, ze za kazdych ziskanych N procent ziska kazda strana jednoho zastupitele, a N zvolime tak, aby byl pocet udelenych zastupitelu spravny (toto je pozdeji zminovana d'Hondtova metoda).
Pokud N=9,5\%, pak strany A, B a C dostanou po dvou zastupitelich a strany D a E po jednom.
Zisky stran z hlediska poslancu jsou tedy 25\% pro A az C a 12.5\% pro D a E.
Procenta sice vysledkum voleb presne neodpovidaji, ale jde o mnohem lepsi reprezentaci nez u vetsinoveho vysledku kde vsechny zastupitele ziskalo A.
Pri volbe vice poslancu by vysledky odpovidaly jeste presneji, lze si vsimnout ze jednoho ani pet poslancu nelze \uv{ferove} rozdelit mezi 5 stran tak aby odpovidali procenta zisku hlasu a zisku mandatu.
Pri rozdelovani stovek poslancu uz pujde jen o male rozdily mezi zminenymi procenty.
Navic, jak ukazal Gallagher, metody rozdelovani (ktere budou vysvetleny nize) pri velkem mnozstvi rozdelovanych mandatu konverguji.\autocite{GAL1}   

\section{Historie}
Matematicka teorie pomerneho rozdeleni mandatu ale saha ke zrodu Spojenych statu Americkych, demokratickeho zrizeni ktere dodnes vyuziva systemu vetsinoveho.\autocite{BAL2}
Protoze vznikaly jako spolecenstvi 13 existujicich statu, bylo jiz od prvnich chvil zasadnim tematem vliv jednotlivych statu ve vznikajicim state, lepe receno pomer vlivu malych a vekych statu.
Pro nas nejzajimavejsim tematem je rozdeleni kresel ve Snemovne reprezenatntu, dolni komore Americkeho zakonodarneho parlamnetu.
Dle americke ustavy prijate 1878 ziskavaji staty representatnty podle poctu obyvatel, svobodny clovek se pocital za jednoho obyvatele, indiani nebyly pocitani vubec a ostatni za 3/5 obyvatele.\autocite{CON}
Ustava ale presny postup deleni nestanovila.
Ve vysledku tedy ziskaly politici spojenych statu stejnou ulohu jako pozdejsi deleni kresel mezi strany dle poctu hlasu, ale slo o deleni kresel statum dle poctu obyvatel. 
Uloha ktery se mohla na prvni pohled zdat az lehka budila kontroverze po dalsich vice nez 150 let.
Byla duvodem uziti vubec prvniho americkeho prezidentskeho veta kdyz Geroge Washington nesouhlasil se zpusobem rozdeleni ktery znevyhodnoval oproti konkurencnimu navrhu jeho rodnou Virginii.\autocite{BAL1}
Az v polovine 20. stoleti doslo k dlouhodobemu ustaleni metody vypoctu.
Ackoliv v Evrope toto tema bylo vzdy spise predmetem akademicke diskuze, 

\chapter{Teoreticky popis metod deleni mandatu}

\section{Pravy pomerovy system} %TODO definovat

\section{Jednotlive metody}
Prestoze vetsina nasledujicich metod byla prvne vymyslena americkymi matematiky, budu pouzivat pro vysvetleni pojmy strana, a hlas a v evrope bezne pouzivane nazvy.
Zaroven predpokladam naprostou ze zadne zisky stran jsou tokove, ze nedojde ke spornym situacim ohledne distribuce krese, ty budou vysvetleny na konci. 

\subsection{Metody Nejvyssich prumeru}
Prvni skupinou metod deleni mandatu jsou takzvane metody nejvyssich prumeru.
Tyto metody funguji tak, ze kazda strana ziskava hlasy podle podilu hlasu ktere obdezela a promene Q.
Tyto podily jsou vsechny zaokrouleny stejnym zpusobem, kazda z metod v teto skupine je charakterizovana svou zaokrouhlovaci metodou.
Q je urceno tak, aby celkovy rozdeleny pocet mandatu byl roven poctu clenu voleneho zastupitelstva.
Dve nejpouzivanejsi metody z teto zkupiny jsou d'Hondtova, ktera vzdy zaokrouhluje dolu a Sainte-Lague, ktera zaokrouhluje dle beznych aritmetickych pravidel zaokrouhlovani.
Tyto metody tvori spektrum od d'Hondtovy metody po Adamsovu, ktera vzdy zaokrouhluje dolu. %TODO Dukaz meznosti Adamse a d'Hondta pro pomerovost
Adamsova metoda garantuje kreslo kazde strane s alespon jednim hlasem, zatimco d'Hondtova metoda garantuje kreslo az za XXX hlasu.

Matematicky lze tyto metody definovat nasledovne:
strany ve volbach jsou ocislovane $1$ az $x$ a deli si $N$ kresel
$h_{x}$ je pocet hlasu strany $x$
$k_{x}$ je pocet ziskanych zastupitelu stranou x
$[]$ je metoda zaokrouhlovani
a
$$k_{x}=[\dfrac{h_{x}}{Q}]$$ kde $Q$ je realne cislo zvolene tak, aby $$N=\sum_{n=1}^{x} (k_{n})$$ 

Q libovolne splnujici podminky vytvori stejny vysledek. %TODO dokazat
Pro d'Hondtovu metodu je $[ ]$ dolni cela cast.
Pro metodu Sainte-Lague je $[ ]$ zaokrouhleni dle aritmetickych pravidel.
Pro Adamsovu metodu je $[ ]$ dolni cela cast.

Pro tyto metody existuje alternativni zpusob vypoctu.
Ten nejprve vysvetlime pro d'Hondtovu metodu.
Pro kazdou stranu je vypocitan podil jejich hlasu postupne se vsemi prirozenymi cisly od 1 do poctu rozdelovanych mandatu.
Tyto pomery kazde strany urcuji jake maximalni Q jim zajisty dany pocet mandatu (prvni pomer strany dava jeden mandat, druhy pomer dva mandaty...).

Pokud tedy je za Q zvoleno N-te nejvyssi cislo mezi vypoctenymi pomery vsech stran, bude Q zvoleno spravne.
Q ale muzeme z vypoctu vypustit uplne, nebot vime ze pokud kazda strana dostane jeden mandat za kazdy jejich vypocteny pomer vyssi nez nebo roven N-temu nejvyssimu cislu, rozdeleno bude N kresel.
Podobne lze vypocitat rozdeleni Sainte Lague pomoci rady $0{,}5, 1{,}5, 2{,}5$, Adams... %TODO

Zajimave je jak se jednotlive systemy chovaji pokud by se trany pred volbami sloucili, v tomto teoretickem pripade tak, ze nova strana ziska soucet hlasu puvodnich stran.
U vyse zmineneho pomeroveho systemu ma samozdrejme strana s vetsim procentnim ziskem vetsi sanci mit v dane oblasti nejvice hlasu.
U metod nejvyssich prumeru je ale situace trochu slozitejsi.
Nejprve se podivejme na d'Hondovu metodu.




\subsection{Metody nejvetsiho zbytku}
Druha skupina metod jsou metody nejvyssiho zbytku.
Tyto metody vyuzivaji takzvne kvoty.
Kvota je pocet hlasu potrebny k zisku jednoho mandatu, za dve kvoty ziska strana dva mandaty...
Kvoty jsou tedy vzdy prirozena cisla.
Kazda strana nejprve ziska pocet mandatu rovny poctu kvot ktere se vejdou do jejiho volebniho zisku.
Diky vyberu kvoty (jak bude dale vysvetleno) muzou byt rozdeleny budto vsechny mandaty, nebo mene nez ma byt.
Pokud je jich rozdeleno mene, ziskavaji zbyle mandaty po jednom strany s nejvetsim zbytkem po deleni poctu ziskanych hlasu kvotou.
Existuji i kvoty ktere nezarucuji ze mandaru nebude rozdeleno vice nez ma, ty pak odebiraji mandaty stranam s nejmensim zbytkem.
Tyto kvoty se ale pouzivaji zridka.
Nejasnou otazkou techto systemu je zpusob urceni kvoty.
Kvota se vzdy urcuje dle celkoveho souctu hlasu a poctu volenych zastupitelu.
Minimalni kvotu lze definovat snadno, nesmi dovolit zisk vice mandatu nez ma byt rozdeleno.
Tato kvota se naziva Drooopova, a je definovana jako prvni prirozene cislo vetsi nez $\dfrac{H}{N+1}$ kde $H$ je soucet vsech hlasu a $N$ pocet rozdelovanych mandatu.
Tuto kvotu lze odvodit nasledovne: Celkove muze byt (bez dorozdelovani dle zbytku) rozdeleno maximalne tolik mandatu, kolikrat se kvota cela vejde do celkoveho poctu hlasu.
Pro kvotu $\dfrac{H}{N+1}$ se kvota vejde do celkoveho poctu hlasu presne $N+1$ krat, pokud se kvota zvysi, uz se do celkoveho poctu hlasu $N+1$ krat nevejde, vejde se tedy maximalne $N$, maximalne muze byt rozdeleno $N$ mandatu.

Druhou pouzivanou kvoto je kvota Hareova, nekdy take nazyvana prirozena kvota. Ta je rovna pomeru celkoveho poctu hlasu a poctu udelovanych mandatu.
Strany s mensim volebnim ziskem vetsinou Ziskaji pri pouziti Hareovy kvoty vice mandatu nez pri uziti Droopovy.  

Slozitejsi otazkou je maximalni hodnota kvoty.
Pro kvotu blizici se nekonecnu by vsechny strany dosltali nejprve nulu a pak $N$ s nejvyssim ziskem po jednom mandatu.

%TODO The Alabama Paradox The Population Paradox The New States Paradox

\section{Volby v krajich}
V mnoha statech se ale rozdelovani dle vyse popsanych metod neprobiha na celostatni urovni, ale oddelene v jednotlivych oblastech.
Smysl tohoto opatreni je, ze obcane maji sve \uv{mistni} zastupce.
Bezne to v praxi funguje tak, ze podle celkoveho poctu hlasu odevzdanych v kazde oblasti, se celkovy pocet kresel nejprve rozdeli mezi jednotlive oblasti.
V kazde z techto oblasti je prideleny pocet mandatu rozdelovan oddelene dle vyse rozebranych pravidel.
Nektere staty rozdelovani neprovadi naprosto oddelene, ale castecne oddelene.
Prikladem muze byt nize vysvetleny system kterym se v CR volil parlament az do roku 2000. % zdroj: zakon
Toto rozdelovani v oblastech side dava volicum \uv{mistni} zastupce, ma ale zasadni nevyhodu.
Male oblasti vedou k naruseni pomerovosti systemu.
V extremnim pripade oblasti kde je volen vzdy jeden zastupce jde o vyse popsany vetsinovy system. 
Ale i vetsi oblasti pomerovost narusi.
Pokud je v oblasti rozdelovano deset kresel, pripada kazde na asi deset procent hlasu.
Pro stranu se ziskem sedm procent je pak
%TODO Porovnani metod na volbach s mnoha mandaty > podobnost, s malo mandaty > znacne rozdily

\chapter{Volebni systemy v praxi}
\section{Vyvoj volebniho systemu v CR}
% Zdroj, volebni zakon https://www.zakonyprolidi.cz/cs/1995-247
Volebnni system Ceske republiky s vyjimkou zavedeni prime volby prezidenta republiky prosel posledni zmenou kolem roku 2000.
Do te byly mandaty ve snemovnich volbach rozdelovany tak, ze kazdemu z 8 volebnich kraju byl udelen pocet mandatu k rozdeleni podle celkoveho poctu hlasu v danem kraji.
V techto krajich pak byl byly tyto mandaty rozdeleny podle Hagenbach-Bishoffovy kvoty (kvota nezarucujici maximalne N rozdelenych mandatu).
Pokud nebyly nejake mandaty rozdeleny rovnou, ale mely byt dorozdeleny podle zbytku, k dorozdeleni nedoslo.
Celostatne se pak secetly zbytky kazde strany a nerozdelene mandaty a doslo k druhemu rozdelovani.
Zde jiz bylo provedeno i dorozdeleni dle zbytku.
Toto dorozdelovani melo ten effekt, ze vyrovnavallo %TODO  
V roce 2000 pak ODS a CSSD dohromady vytvorili novy system.
Pocet volebnich kraju mel byt zvysen na 35.
Vsechno rozdelovani melo byt provedeno v krajich.
Zpusob rozdeleni mandatu mezi kraje byl zachovan.
Nove mela byt vyuzita k rozdeleni modifikovana metoda d'Hondtova.
Modifikace spocivala v tom, ze pri rozdelovani pomoci postupneho deleni radou zacinala rada cislem 1,42.
Z hlediska popisu d'Hondtovy metody pomoci Q bylo zaokrouhlovani upraveno tak, ze cisla do 1,42 se zaokrouhlovala na 0, od 1,42 do 2 (mimo) na 1 dale pak bezne dolu, od 2 do 3 (mimo) na 2 atd.
System nebyl ciste pomerovy.
Tento system byl ale ustavnim soudem zrusen, nebot se prilis blizil vetsinovemu systemu. \autocite{SOU}
Byla zrusena uprava d'Hondtovy metody a volebni kraje byly sjednoceny se samospravnimy kraji.
Tato uprava vydrzela dodnes.

\section{Pomerovost soucasneho systemu}
V teto casti budeme zkoumat vysledky voleb z roku 2006. 
%zdroj dat?
Tyto volby totiz ukazaly nektere zvastni vlastnosti ceskeho volebniho systemu.
Vysledky voleb byly nasledujici:

\begin{center}
\catcode`\-=12
\begin{tabular}{|c|c|c|c|c|c|c|c|}
\hline
\multirow{2}{*}{Strana}  & \multicolumn{2}{|c|}{Skutecne Vysledky} & \multirow{2}{*}{Teoreticke mandaty} & \multicolumn{4}{|c|}{Rozdeleni dle ruznych metod} \\ \cline{2-3} \cline{5-8}
& Hlasy & Mandaty & & d'Hondt & Adams & Hare & Droop \\  
\hline
ODS & 1892475 & 81 & 75.25 & 76 & 75 & 75 & 76 \\ 
\hline
ČSSD & 1728827 & 74 & 68.74 & 69 & 68 & 69 & 69 \\ 
\hline
SZ & 336487 & 6 & 13.38 & 13 & 14 & 14 & 13 \\ 
\hline
KSČM & 685328 & 26 & 27.25 & 27 & 27 & 27 & 27 \\ 
\hline
KDU-ČSL & 386706 & 13 & 15.38 & 15 & 16 & 15 & 15 \\ 
\hline
\end{tabular} 
\end{center}
%TODO volebni pat?
Jak je nazorne vydet rozdeleni mandatu jak ho preovedl cesky volebni system neodpovida poctu obdrzenych hlasu.
Tento rozdil je zpusoben vyse zminenym rozdelovanim mandatu v oblastech.
V soucasnem ceskem systemu je oblasti kazdy samospravny kraj.
Lide v kazdem kraji voli sve poslance za \uv{svuj}kraj.
Pro prozkoumani vyse zminenych vysledku tedy volby dvakrat prepocitame, jednou tak, jako kdyby republiku tvorilo 8 kraju o 25 mandatech (volebne nejvetsi kraj Praha) a kdyby ho tvorilo 40 kraju o 5 mandatech (volebne nejmensi kraj Karlovarsky).
Budeme predpokladat ze v kazdem kraji je pomer hlasu stejny jako celostatne.



2 \& 80\\
2 \& 80\\
0 \& 0\\
1 \& 40\\
0 \& 0\\
\\
1\\
1\\
1\\
1\\
1\\


10\& 80 \\
9 \& 72\\
1 \& 8\\
3 \& 24\\
2 \& 16\\

9\\
8\\
2\\
4\\
2\\

Kdyz porovname tyto vysledky s vysledky celostatnimi, je videt ze deleni republiky na kraje znamena, ze zpusob rozdelovani mandatu haje zasadni roli.
Pri celostatnim vyhodnoceni se metody lisi maximalne o jeden madnat, narozdil od propastnych rozdilu pri deleni v krajich.
Zde vznioka vyse nelogicky vypadajici mezi ziskem mandatu teoretickym na celostatni urovni a slozenim parlamentu jak ho vytvoril volebni system.

\chapter{Mozne upravy a zaver}

\section{Celostatni rozdelovani}
Slovensko se vyse zminenych problemu zbavilo tak, ze jejich volby probihaji celostatne.
Jak bylo videt vyse, pri celostatnim rozdelovani mandatu je volba metody jiz otazkou nekolika malo mandatu.
Nevyhoda pro volice je, ze kazdy dostava kandidatni listinu o 150 kandidatech (i kdyz to je dano spise optimismem politickych stran nez podpbou systemu) a nema \uv{sve} poslance.
I kdydz toto muzeme povazovat za reseni problemu, podivejme se, jestlki by bylo mozne zachovat oblastni rozdelovani mandatu.
\section{Jiny zpusob deleni mandatu v krajich}
Jak bylo zmineno vyse  %potreba pridat do teoretickeho uvodu
d'Hondtova metoda neni povazovana za nejvice pomerovou.
SL je v tomto ohlodu hodnocena jako jedna z nejlepsich metod % zdroje
takze moznym zlepsenim situace je zavedeni teto metody misto metody d'Hondtovy.
Problemem je ze v krajich jako je Karlovarsky neni rozdeleni ktere by presne odpovidalo rozdeleni hlasu casto vubec mozne.
Strana s 10 procenty hlasu nemuze dostat nic blizsiho svemu zisku nez 0 nebo 20 procent kresel.
Pohou zmenou metody tedy problem resit nelze. 
\section{NUTS 2 kraje}
Jakl bylo vyse popsane, cim mensi kraj tim tezsi je v nem dosahnout pomerneho rozdeleni.
Dalsi moznosti uporavy systemu je tedy zvetseni volebnich oblasti.
Vytvoreni oblasti ktere by slouzili jen pro volby by ale melo nekolik problemu.
Prvni z neich je ze se ztraci puvodni vyznam voleb v krajich, tedy to aby volice zastupovali \uv{jejich} mistni zastupci s kterymi se ztotozkuji.
Umele rozdeleni republiky na \uv{volebni kraje} muze ale take vest k politickym bojum o jejich hranice a pocet. % zdroj o gerrymandering
Je proto praktictejsi aby se volebni kraje shodovali s jiz existujicimi hranicemi.
Nabizi se pouzit NUTS 2 regiony, jak je uzivaji Evropske i Ceske instituce vcetne Ceskeho statictickeho uradu.
%https://www.czso.cz/documents/11244/43384376/NUTS2_CR_2015.pdf/8f87a03e-bd06-4bd1-9fbe-ae6578b5ede4?redirect=https%3A%2F%2Fwww.czso.cz%2Fcsu%2Fczso%2Fcri%2Fpohyb-obyvatelstva-4-ctvrtleti-2015%3Fp_p_id%3D3%26p_p_lifecycle%3D0%26p_p_state%3Dmaximized%26p_p_mode%3Dview%26_3_groupId%3D0%26_3_keywords%3Dnuts%26_3_struts_action%3D%252Fsearch%252Fsearch%26_3_redirect%3D%252Fweb%252Fczso%252Fkatalog-produktu-vydavame
Kazdy z techto regionu je tvoren jednim nebo vice kraji.
Je jich 8 a kazdy z nich ma vice nez 1,1 milionu obyvatel.
Ani toto ale neresi vsechny problemy.
Jak bylo ukazano vyse ani regiony velikosti Prahy (Praha je jednim z NUTS 2 regionu), nejsou zarukou pomerovosti.
Takto velike regiony by se ale pri spravne zvolene metode deleni mansatu mohly pomerovosti blizit, ale stale jsou jeste prilis male na to aby presne odpovidaly poctum hlasu.
Navic zde opet narazime na neplneni puvodniho umysl, vytvoreni oblasti se kterymi volici citi spojenost.
\section{Rozdelovani mandatu stran do kraju}
Reseni ktere by autor teto prace chtel predlozit je rozdelit mandaty na celostatni mandaty a ziskat tak rozlozeni sil v parlamentu ktere by presne odpovidalo pomerum hlasu ktere byly odevzdany ve volbach.
Mandaty kazde strany by se pak delili mezi kandidaty v jednotlivych krajich podle poctu hlasu pro strany v jednotlivych krajich.
Tento system by zarucoval ze hlasy stranam ktere se dostanou do parlamentu by v zadnem kraji nebyly irrelevantni, jak tomu ted je u stran s malym ziskem v malych krajich. 
Zachovala by se prislusnost zvolenych poslancu krajum.
Tento postup ma ale take sve nevyhody.
Problem deleni maleho poctu mandatu mezi strany se promenil mezi deleni maleho poctu mandatu mezi kraje.
To uz ale neovlivnuje vysledek voleb tak jak ho zname, tedy rodeleni kresel stranam.

%TODO metoda deleni mandatu mezi kraje
%TODO vysledky spocitane navrhovanym systemem

\section{Zaver}

\section{Reference}
\printbibliography[heading=none]

\end{document}
