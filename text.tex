\documentclass[12pt,a4paper]{article}
\usepackage[utf8]{inputenc}
\usepackage[czech]{babel}
\usepackage[T1]{fontenc}
\usepackage{amsmath}
\usepackage{hyperref}
\usepackage{amsfonts}
\usepackage{amssymb}
\usepackage{graphicx}
\usepackage{lmodern}
%\usepackage{csquotes}
\usepackage[backend=biber, style=numeric, citestyle=numeric]{biblatex}
\addbibresource{citace.bib}
\usepackage{indentfirst}
\linespread{1.5}
\usepackage[left=2.5cm,right=2cm,top=2.5cm,bottom=2cm]{geometry}
\author{David Zacek}
\title{Volby a matematika}
\begin{document}
\section{Uvod}
Cilem teto prace je analyzovat soucasny volebni system v CR a navrhnout jeho upravy, ktere by ho udelali spravedlivejsim. 
Volebnim systemem budu v teto praci nazyvat postup, dle ktereho je mozne mezi strany ucastnici se voleb rozdelit na zaklade poctu hlasu ktere obdrzely nedelitelne mandaty ve volenem sboru.
Tyto systemu mohou ale byt analogicky pouzity i v dalsich situacich, nebot zobecnene jde o rozdeleni k nedelitelnych objektu mezi n subjeku, s tim ze kazdy subjekt ma ciselne ohodnoceni a rozdeleni by melo byt provedeno tak, aby pomer obdrzenych objeku a vlastniho ohodnoceni kazdeho subjektu byl co nejpodobnejsi. 

\section{Deleni volebnich systemu}
Volebni systemy lze delit na vetsinove a pomerove. Vetsinove systemy jsou ty, ktere se ptaji pouze na otazku, kdo ma nejvic a jsou nezavisle na poctech hlasu dalsich kandidatu.
Vitezi pouze ten s nejvetsim poctem hlasu. Pro volbu vetsiho poctu osob se zpravila voli kazda disjunktni skupina obyvatel sveho zastupce. Prikladem vetsinoveho systemu jsou parlamentni volby v UK a USA, kde kazdou oblast reprezentuje ten kandidat, ktery ve volbach ziskal nejvice hlasu. Tento zpusob volby odpovida puvodnimu pojeti demokracie, jak ho zname jiz z antiky. 
Z moderniho pohledu ma ale tento pristup zasadni vadu, zcela ignoruje velke mnozstvi hlasu.
To lze demonstrovat na hypotetickem na prikladu. Ve volbach se o prizen volicu uchazi pet nazoru na fungovani zeme, ktere budu oznacovat A az E.
Celkem se voli 8 zastupielu.
Kazdy nazor ma sve poporovatele, nazor A podporuje 25\% obyvatel, nazor B 21\%, C 19\%, D 18\% a E 17\%.
Vsechny nazory jsou u vsech skupinach obyvatel ve stejnem pomeru, a predpokladame dokonalou spolupraci vsech priznivcu kazdeho nazoru a neochotu pracovat se stoupenci jineho nazoru.
Pokud by jsme vytvorili 8 skupin obyvatel (napriklad podle bydliste) a kazdou nechali rozhodnout o jednom zastupiteli, vsech 8 zastupitelu by nakonec podporovalo nazor A.
To proto, ze A by v kazdem hlasovani navrhlo jedineho kandidata, ktery by s 25\% podporou zisal vic hlasu nez libovolny kandidat dalsi strany. 
Tento vysledek se ale nezda velmi spravedlivy z toho pohledu, ze proti nazoru A hlasovaly tri ctvrtiny obyvatel a presto ziskali jeho stoupenci vsech 8 zastupitelskych pozic.
Pokud by se v celem state hlasovalo o stejnych kandidatech a tech 5 kandidatu s nejvice hlasy by bylo zvoleno, tak by strategie z hlediska poctu kandidatu kazdeho nazoru mohla byt rozhodujuci faktor voleb. 

Pomerove zastoupeni tento problem resi.
Namisto otazky kdo ziskal nejvice se pta ko ziskal jakou cast. Lide se stejnym nazorem pak do voleb vstupuji jako kandidujici politicka strana, lide hlasuji pro strany a kazda strana pak dostane tolik zaptupitelu, aby jeji procentualni zisk ve volbach co nejvice odpovidal zisku zastupitelu.
Otazka jak provest rozdeleni mandatu dle vysledku voleb bude hlavnim predmetem zkoumani teto prace.
Nyni muzeme pro predstavu udelit zastupitele stranam (odpovidajicim nazorum) A az E napr tak, ze za kazdych ziskanych N procent ziskaji jednoho zastupitele, a N zvolime tak, aby byl pocet udelenych zastupitelu spravny (toto je pozdeji zminovana d'Hondtova metoda).
Pokud N=9,5\%, pak strany A, B a C dostanou po dvou zastupitelich a strany D a E po jednom.
Zisky stran z hlediska zastupitelu jsou tedy 25\% pro A az C a 12.5\% pro D a E.
Procenta sice s vysledky voleb presne neodpovidaji, ale jde o mnohem lepsi reprezentaci nez u vetsinoveho vysledku kde vsechny zastupitele ziskalo A.
Zde je videt jedna z vlastnosti pomeroveho zastoupeni, cim vice zastupitelu se voli, tim lepe mohou vysledky pri rozdfelovani zastupitelstva odpovidat nazorum obyvatelstva.
Pokud pro dane rozlozeni sil budeme volit nasledujici pocty zastupitelu, budou odchylky od \uv{spravneho} zisku:

\section{Pravy pomerovy system} %TODO definovat

\section{Historie}
Matematicka teorie pomerneho rozdeleni mandatu ale saha ke zrodu demokratickeho zrizeni ktere dodnes vyuziva systemu vetsinoveho.
Na konci 18.
stoleti vznikaly spojene staty americke.\autocite{BAL2}
Protoze vznikaly jako spolecenstvi 13 existujicich statu, bylo jiz od prvnich chvil zasadnim tematem vliv jednotlivych statu ve vznikajicim state, lepe receno pomer vlivu malych a vekych statu.
Ze snah vyrovnat vliv vychazi mnoho prvku americkeho politickeho systemu ktere prezivaji dodnes, napriklad Sbor volitelu, 2 clenove senaru pro kazdy stat... 
Pro nas nejzajimavejsim tematem je rozdeleni kresel ve Snemovne reprezenatntu, dolni komore americkeho zakonodarneho Congressu.
Dle americke ustavy prijate 1878 ziskavaji staty representatnty podle poctu obyvatel.
Svobodny clovek se pocital za jednoho obyvatele, otrok za 3/5 obyvatele a indiani nebyly pocitani vubec.
Ustava ale presny postup deleni nestanovila.
Ve vysledku tedy ziskaly politici spojenych statu stejnou ulohu jako pozdejsi deleni kresel mezi strany dle poctu hlasu, ale slo o deleni kresel statum dle poctu obyvatel. 
Uloha ktery se mohla na prvni pohled zdat az trivialne lehka budila kontroverze po dalsich vice nez 150 let.
Byla duvodem uziti vubec prvniho americkeho prezidentskeho veta kdyz Geroge Washington nesouhlasil se zpusobem rozdeleni ktery znevyhodnoval oproti konkurencnimu navrhu jeho rodnou Virginii.
Az v polovine 20. stoleti doslo k dlouhodobemu ustaleni metody vypoctu.
Mezitim ale bylo pouzivano hned nekolik ruznych metod deleni. %TODO

\section{Jednotlive metody}
Prestoze vetsina nasledujicich metod byla prvne vymyslena americkymi matematiky, budu pouzivat pro vysvetleni pojmy strana, a hlas a v evrope bezne pouzivane nazvy.
Zaroven predpokladam naprostou ze zadne zisky stran jsou tokove, ze nedojde ke spornym situacim ohledne distribuce krese, ty budou vysvetleny na konci. 

\subsection{Metody Nejvyssich prumeru}
Prvni skupinou metod deleni mandatu jsou takzvane metody nejvyssich prumeru.
Tyto metody funguji tak, ze kazda strana ziskava hlasy podle podilu hlasu ktere obdezela a nezname Q.
Tyto podily jou vsechny zaokrouleny stejnym zpusobem, kazda z metod v teto skupine je charakterizovana svou zaokrouhlovaci metodou.
Q je urceno tak, aby celkovy rozdeleny pocet mandatu byl roven poctu clenu voleneho zastupitelstva.
Dve nejpouzivanejsi metody z teto zkupiny jsou d'Hondtova, ktera vzdy zaokrouhluje dolu a Sainte-Lague, ktera zaokrouhluje dle beznych pravidel zaokrouhlovani.
Tyto metody tvori spektrum od d'Hondtovy po Adamsovu, ktera vzdy zaokrouhluje dolu.
Adamsova metoda garantuje kreslo kazde strane s alespon jednim hlasem, zatimco d'Hondtova metoda garantuje kreslo az za XXX hlasu.

Matematicky lze tyto metody definovat nasledovne:
strany ve volbach jsou ocislovane $1$ az $x$ a deli si $N$ kresel
$h_{x}$ je pocet hlasu strany $x$
$k_{x}$ je pocet ziskanych zastupitelu stranou x
$[]$ je metoda zaokrouhlovani
a
$$k_{x}=[\dfrac{h_{x}}{Q}]$$ kde $Q$ je realne cislo zvolene tak, aby $$N=\sum_{n=1}^{x} (k_{n})$$ 

Q libovolne splnujici podminky vytvori stejny vysledek. %TODO dokazat
Pro d'Hondtovu metodu je $[]$ dolni cela cast.
Pro metodu Sainte-Lague je $[]$ zaokrouhleni dle aritmetickych pravidel.
Pro Adamsovu metodu je $[]$ dolni cela cast.

Pro tyto metody existuje alternativni zpusob vypoctu.
Ten nejprve vysvetlime pro d'Hondtovu metodu.
Pro kazdou stranu je vypocitan podil jejich hlasu postupne se vsemi prirozenymi cisly od 1 do poctu rozdelovanych mandatu.
Tyto pomery kazde strany urcuji jake maximalni Q jim zajisty dany pocet mandatu (prvni pomer strany dava jeden mandat, druhy pomer dva mandaty...).
Pokud tedy je za Q zvoleno N-te nejvyssi cislo mezi pomery vsech stran, bude Q zvoleno spravne.
Q ale muzeme z vypoctu vypustit uplne, nebot vime ze pokud kazda strana dostane jeden mandat za kazdy jejich vypocteny pomer vyssi nez N-te nejvyssi cislo, rozdeleno bude N-1 kresel.
Pak strana ktera ma Q mezi svymi pomery ziska posledni N-te kreslo.
Podobne lze vypocitat rozdeleni Sainte Lague pomoci rady $0{,}5, 1{,}5, 2{,}5$, Adams... %TODO

\subsection{Metody nejvetsiho zbytku}
Druha skupina metod jsou metody nejvyssiho zbytku.
Tyto metody vyuzivaji takzvne kvoty.
Kvota je pocet hlasu potrebny k zisku jednoho mandatu, za dve kvoty ziska strana dva mandaty...
Kvoty jsou tedy vzdy prirozena cisla.
Kazda strana nejprve ziska pocet mandatu rovny poctu kvot ktere se vejdou do jejiho volebniho zisku.
Diky vyberu kvoty (jak bude dale vysvetleno) muzou byt rozdeleny budto vsechny mandaty, nebo mene nez ma byt.
Pokud je jich rozdeleno mene, ziskavaji zbyle mandaty po jednom strany s nejvetsim zbytkem po deleni poctu ziskanych hlasu kvotou.
Existuji i kvoty ktere nezarucuji ze mandaru nebude rozdeleno vice nez ma, ty pak odebiraji mandaty stranam s nejmensim zbytkem.
Tyto kvoty se ale pouzivaji zridka.
Nejasnoo otazkou techto systemu je zpusob urceni kvoty.
Kvota se vzdy urcuje dle celkoveho souctu hlasu a poctu volenych zastupitelu.
Minimalni kvotu lze definovat snadno, nesmi dovolit zisk vice mandatu nez ma byt rozdeleno.
Tato kvota se naziva Drooopova, a je definovana jako prvni prirozene cislo vetsi nez $\dfrac{H}{N+1}$ kde $H$ je soucet vsech hlasu a $N$ pocet rozdelovanych mandatu.
Tuto kvotu lze odvodit nasledovne: Celkove muze byt (bez dorozdelovani dle zbytku) rozdeleno maximalne tolik mandatu, kolikrat se kvota cela vejde do celkoveho poctu hlasu.
Pro kvotu $\dfrac{H}{N+1}$ se kvota vejde do celkoveho poctu hlasu presne $N+1$ krat, pokud se kvota zvysi, uz se do celkoveho poctu hlasu $N+1$ krat nevejde, vejde se tedy maximalne $N$, maximalne muze byt rozdeleno $N$ mandatu.

Slozitejsi otazkou je maximalni hodnota kvoty.
Pro kvotu blizici se nekonecnu by vsechny strany dosltali nejprve nulu a pak $N$ s nejvyssim ziskem po jednom mandatu.

\section{Vyvoj volebniho systemu v CR}
Volebnni system Ceske republiky s vyjimkou zavedeni prime volby prezidenta republiky prosel posledni zmenou kolem roku 2000.
Do te byly mandaty ve snemovnich volbach rozdelovany tak, ze kazdemu z 8 volebnich kraju byl udelen pocet mandatu k rozdeleni podle celkoveho poctu hlasu v danem kraji.
V techto krajich pak byl byly tyto mandaty rozdeleny podle Hagenbach-Bishoffovy kvoty (kvota nezarucujici maximalne N rozdelenych mandatu).
Pokud nebyly nejake mandaty rozdeleny rovnou, ale mely byt dorozdeleny podle zbytku, k dorozdeleni nedoslo.
Celostatne se pak secetly zbytky kazde strany a nerozdelene mandaty a doslo k druhemu rozdelovani.
Zde jiz bylo provedeno i dorozdeleni dle zbytku.
Toto dorozdelovani melo ten effekt, ze vyrovnavallo %TODO  
V roce 2000 pak ODS a CSSD dohromady vytvorili novy system.
Pocet volebnich kraju mel byt zvysen na 35.
Vsechno rozdelovani melo byt provedeno v krajich.
Zpusob rozdeleni mandatu mezi kraje byl zachovan.
Nove mela byt vyuzita k rozdeleni modifikovana metoda d'Hondtova.
Modifikace spocivala v tom, ze pri rozdelovani pomoci postupneho deleni radou zacinala rada cislem 1,42.
Z hlediska popisu d'Hondtovy metody pomoci Q bylo zaokrouhlovani upraveno tak, ze cisla do 1,42 se zaokrouhlovala na 0, od 1,42 do 2 (mimo) na 1 dale pak bezne dolu, od 2 do 3 (mimo) na 2 atd.
System nebyl ciste pomerovy.
Tento system byl ale ustavnim soudem zrusen, nebot se prilis blizil vetsinovemu systemu. \autocite{SOU}
Byla zrusena uprava d'Hondtovy metody a volebni kraje byly sjednoceny se samospravnimy kraji.
Tato uprava vydrzela dodnes.

\section{Pomerovost soucasneho systemu}
      
\printbibliography 

\end{document}
