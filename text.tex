\documentclass[12pt,a4paper]{scrreprt}
\usepackage[utf8]{inputenc}
\usepackage[british,czech]{babel}
\usepackage[T1]{fontenc}
\usepackage{amsmath}
\usepackage{hyperref}
\usepackage{amsfonts}
\usepackage{amssymb}
\usepackage{graphicx}
\usepackage{lmodern}
\usepackage[hang,flushmargin]{footmisc} 
\usepackage{multirow}
\usepackage{chngcntr}

\usepackage[backend=biber,language=british, style=numeric, citestyle=numeric]{biblatex}
\addbibresource{citace.bib}
\usepackage{indentfirst}
\usepackage{setspace}
\onehalfspacing
\usepackage[left=2.5cm,right=2cm,top=2.5cm,bottom=2cm]{geometry}
\makeindex
\begin{document}
\begin{titlepage}
	\begin{center}
	{\large Gymnázium Christiana Dopplera, Zborovská 45, Praha 5 \par}
	\vfill
	\par\vspace{1cm}
	{\scshape\LARGE ROČNÍKOVÁ PRÁCE \par}
	{\huge\bfseries Analyza systemu voleb do Poslanecke snemovny Ceske republiky\par}
	\vfill
\end{center}
Vypracoval: David Žáček \\
Třída: 8. M \\
Školní rok: 2016/2017 \\
Seminář : Matematický seminář \\
	\vfill
\end{titlepage}

\vspace*{\fill}
\begin{quote}
Prohlašuji, že jsem svou ročníkovou práci napsal samostatně a výhradně s
použitím 
citovaných pramenů. Souhlasím s
využíváním 
práce na Gymnáziu Christiana Dopplera 
pro studijní účely. \\
\end{quote}

V Praze dne \today \hfill David Žáček

\vspace*{\fill}
\thispagestyle{empty}
\newpage
\tableofcontents
\newpage
\chapter{Uvod}
\section{Cile}
Cilem teto prace je analyzovat soucasny volebni system  pro volby do Poslanecke snemovny Parlamentu Ceske republiky a navrhnout jeho upravy, ktere by ho udelali spravedlivejsim. 
Volebnim systemem budu v teto praci nazyvat postup, dle ktereho se mezi strany ucastnici se voleb rozdeluji mandaty na zaklade poctu hlasu ktere obdrzely. 
Nebude zde zkoumana umela hranice pro zahrnuti strany do rozdelovani mandatu, u nas znama jako \uv{petiprocentni hranice}.
\footnote{Petiprocentni hranice ve skutecnosti neni zcela presny nazev. Hranice je jednak vyssi pro koalice stran, atake muze byt snizena pokud by ji neprekrocili alespon dve strany.\autocite{ZAK}}
Tato hranice existuje ve vetsine statu ktere vyuzivaji pomerovy system a byla predmetem mnoha zkoumani a teto prace se bude venovat jinym otazkam.
Hlasy pro strany ktere hranice nedosahly budou v praci povazovany za neodevzdane ci neplatne.
I kdyz budou zkoumany volby v CR, mohou zavery byt aplikovany i v jinych statech. 
\section{Deleni volebnich systemu}
Volebni systemy lze delit na vetsinove a pomerove. Vetsinove systemy jsou ty, ktere se ptaji pouze na otazku, kdo ma nejvic a jsou nezavisle na poctech hlasu dalsich kandidatu.
Vitezi pouze ten s nejvetsim poctem hlasu.
Pro volbu vetsiho poctu osob se zpravila voli kazda disjunktni skupina obyvatel sveho jednoho zastupce.
\footnote{Existuje i moznost aby zvoleno bylo v oblasti vice kandidatu s nejvice hlasy, to se deje napriklad pri volbach ve state Tuvalu.\autocite{TUV}}
Prikladem vetsinoveho systemu jsou parlamentni volby v UK a USA, kde kazdou oblast reprezentuje ten kandidat, ktery v ni ziskal nejvice hlasu.
Tento postup ma ale pomerne zasadni vadu, zcela ignoruje velke mnozstvi hlasu.
To lze demonstrovat na hypotetickem na prikladu. Ve volbach se o prizen volicu uchazi pet nazoru na fungovani zeme, ktere budu oznacovat A az E.
Celkem se voli 8 poslancu.
Kazdy nazor ma sve poporovatele, nazor A podporuje 25\% obyvatel, nazor B 21\%, C 19\%, D 18\% a E 17\%.
Predpokladejme, ze vsechny nazory jsou u vsech skupinach obyvatel ve stejnem pomeru, a predpokladame dokonalou spolupraci vsech priznivcu kazdeho nazoru a neochotu pracovat se stoupenci jineho nazoru.
Pokud bysme vytvorili 8 skupin obyvatel (napriklad podle bydliste) a kazdou nechali rozhodnout o jednom poslanci, vsech 8 poslancu by nakonec podporovalo nazor A.
To proto, ze A by v kazdem hlasovani navrhlo jedineho kandidata, ktery by s 25\% podporou zisal vic hlasu nez libovolny kandidat dalsi strany. 
Tento vysledek se ale nezda velmi spravedlivy z toho pohledu, ze proti nazoru A hlasovaly tri ctvrtiny obyvatel a presto ziskali jeho stoupenci vsechna kresla v parlamentu.

Pomerove zastoupeni tento problem resi.
Namisto otazky kdo ziskal nejvice se pta, kdo ziskal jakou cast. Kandidati se stejnym nazorem pak do voleb vstupuji jako kandidujici politicka strana, volici hlasuji pro strany a kazda strana pak dostane tolik zaptupitelu, aby jeji procentualni zisk ve volbach co nejvice odpovidal zisku zastupitelu.
Otazka jak provest rozdeleni mandatu dle vysledku voleb bude hlavnim predmetem zkoumani teto prace.
Nyni muzeme pro predstavu udelit poslance stranam (odpovidajicim nazorum) A az E napr tak, ze za kazdych ziskanych N procent ziska kazda strana jednoho zastupitele, a N zvolime tak, aby byl pocet udelenych zastupitelu spravny (toto je pozdeji zminovana d'Hondtova metoda).
Pokud N=9,5\%, pak strany A, B a C dostanou po dvou zastupitelich a strany D a E po jednom.
Zisky stran z hlediska poslancu jsou tedy 25\% pro A az C a 12.5\% pro D a E.
Procenta sice vysledkum voleb presne neodpovidaji, ale jde o mnohem lepsi reprezentaci nez u vetsinoveho vysledku kde vsechny zastupitele ziskalo A.
Pri volbe vice poslancu by vysledky odpovidaly jeste presneji, lze si vsimnout ze jednoho ani pet poslancu nelze \uv{ferove} rozdelit mezi 5 stran tak aby odpovidali procenta zisku hlasu a zisku mandatu.
Pri rozdelovani stovek poslancu uz pujde jen o male rozdily mezi zminenymi procenty.
Navic, jak ukazal Gallagher, metody rozdelovani (ktere budou vysvetleny nize) pri velkem mnozstvi rozdelovanych mandatu konverguji.\autocite{GAL1}
To jak se procento ziskanych mandatu  blizi procentu ziskanych hlasu bude dale v praci nazyvano pomerovosti systemu.
I kdyz existuji numericke indexy pomerovosti, lisi se stale nazor na jejich vyznam a tak be v praci vyuzita intuitivni definice pomerovosti.\autocite{GAL2}

\section{Historie}
Matematicka teorie pomerneho rozdeleni mandatu ale saha ke zrodu Spojenych statu Americkych, demokratickeho zrizeni ktere dodnes vyuziva systemu vetsinoveho.\autocite{BAL2}
Protoze vznikaly jako spolecenstvi 13 existujicich statu, bylo jiz od prvnich chvil zasadnim tematem vliv jednotlivych statu ve vznikajicim state, lepe receno pomer vlivu malych a vekych statu.
Pro nas nejzajimavejsim tematem je rozdeleni kresel ve Snemovne reprezenatntu, dolni komore Americkeho zakonodarneho parlamnetu.
Dle americke ustavy prijate 1878 ziskavaji staty representatnty podle poctu obyvatel, svobodny clovek se pocital za jednoho obyvatele, indiani nebyly pocitani vubec a ostatni za 3/5 obyvatele.\autocite{CON}
Ustava ale presny postup deleni nestanovila.
Ve vysledku tedy ziskaly politici spojenych statu stejnou ulohu jako pozdejsi deleni kresel mezi strany dle poctu hlasu, ale slo o deleni kresel statum dle poctu obyvatel. 
Uloha ktery se mohla na prvni pohled zdat az lehka budila kontroverze po dalsich vice nez 150 let.
Byla duvodem uziti vubec prvniho americkeho prezidentskeho veta kdyz Geroge Washington nesouhlasil se zpusobem rozdeleni ktery znevyhodnoval oproti konkurencnimu navrhu jeho rodnou Virginii.\autocite{BAL1}
Az v polovine 20. stoleti doslo k dlouhodobemu ustaleni metody vypoctu.
Jak uvadi Carstair, v Evrope se tematu zacina venovat pozornost az na konci 19. stoleti.
V roce 1885 probiha v Antverpach konference na tema pomeroveho volebniho systemu.
V nasledujicich letech zacnou stty zapadni Evropy postupne tento ovy system vyuzivat, zde jiz ale nedochazi k takovym kontroverzim.
Do roku 1920 uz vyuziva pomeroveho systemu vetsina zapadni Evropy.\autocite{BOO}

\chapter{Teoreticky popis metod deleni mandatu}
\section{Jednotlive metody pomeroveho zastoupeni}
Metodou pomeroveho zastoupeni budu nazyvat metodu ktera rozdeli mandaty mezi politicke strany na zaklade poctu ziskanych hlasu, at to tak, ze se snazi dosahnou pomerovosti.
Prestoze vetsina nasledujicich metod byla prvne vymyslena americkymi matematiky, budu pouzivat pro vysvetleni pojmy strana a hlas a v Evrope bezne pouzivane nazvy.
Zaroven pri kazdem porovnavani dvou cisel predpokladame ostrou nerovnost, metody reseni rovnosti budou popsany na konci teto kapitoly.
\subsection{Metody nejvyssich prumeru}
Prvni skupinou metod deleni mandatu jsou takzvane metody nejvyssich prumeru.
Tyto metody funguji tak, ze kazda strana ziskava hlasy podle podilu hlasu ktere obdezela a promene Q.
Tyto podily jsou vsechny zaokrouleny stejnym zpusobem, kazda z metod v teto skupine je charakterizovana svou zaokrouhlovaci metodou.
Q je urceno tak, aby celkovy rozdeleny pocet mandatu byl roven poctu clenu voleneho zastupitelstva.
Dve nejpouzivanejsi metody z teto zkupiny jsou d'Hondtova, ktera vzdy zaokrouhluje dolu a Sainte-Lague, ktera zaokrouhluje dle beznych aritmetickych pravidel zaokrouhlovani.
Tyto metody tvori spektrum od d'Hondtovy metody , ktera vzdy zaokrouhluje dolu, po Adamsovu, ktera vzdy zaokrouhluje nahoru.
Vsechny ostatni metody lezi na spektru protoze nekdy zaokropuhluji nahoru a nekdy dolu.\footnote{Napriklad tvz. Danska metoda zaokrouhluje nahoru, pokud cislo je mene nez $\frac{2}{3}$ pod horni celou mezi cisla. 4.32 je tedy zaokrouhleno na 4, ale 4.34 je zaokrouhleno na 5.} 
Mezi tyto metody je podle Gallaghera chybne nekdy razena Imperaliho metoda.
To funguje jako d'Hondtova metoda ale, od zisku kazde strany odecita jeden mandat.
Jak ukazal, tato metoda ale nehleda pomerove rozdeleni, ale jakesi pseudo-pomerove, ktere ma posilit pozici velkych stran.
\autocite{GAL1}
\newpage
Matematicky lze tyto metody definovat nasledovne:
\begin{itemize}
\item strany ve volbach jsou ocislovane $1$ az $n$ a deli si $N$ kresel
\item $h_{x}$ je pocet hlasu strany $x$
\item $k_{x}$ je pocet ziskanych zastupitelu stranou x
\item $[ ]$ je metoda zaokrouhlovani
\item $k_{x}=[\dfrac{h_{x}}{Q}]$ kde $Q$ je realne cislo zvolene tak, aby patilo: $$N=\sum_{n=1}^{x} (k_{n})$$
\end{itemize}
Q libovolne splnujici podminky vytvori stejny vysledek.
Pro d'Hondtovu metodu je $[ ]$ dolni cela cast.
Pro metodu Sainte-Lague je $[ ]$ zaokrouhleni dle aritmetickych pravidel.
Pro Adamsovu metodu je $[ ]$ dolni cela cast.

Pro tyto metody existuje alternativni zpusob vypoctu.
Ten nejprve vysvetlime pro d'Hondtovu metodu.
Pro kazdou stranu je vypocitan podil jejich hlasu postupne se vsemi prirozenymi cisly od 1 do poctu rozdelovanych mandatu.
%TODO dovysvetlit
Tyto pomery kazde strany urcuji jake maximalni Q jim zajisty dany pocet mandatu (prvni pomer strany dava jeden mandat, druhy pomer dva mandaty...).
Pokud tedy je za Q zvoleno N-te nejvyssi cislo mezi vypoctenymi pomery vsech stran, bude Q zvoleno spravne.
Q ale muzeme z vypoctu vypustit uplne, nebot vime ze pokud kazda strana dostane jeden mandat za kazdy jejich vypocteny pomer vyssi nez nebo roven N-temu nejvyssimu cislu, rozdeleno bude N kresel.
Podobne lze vypocitat rozdeleni Sainte-Lague pomoci rady $0{,}5, 1{,}5, 2{,}5...$.
Pomery v rade kterou je pocet hlasu delen odpovidaji hranici zaokrouhleni.

\subsection{Metody nejvetsiho zbytku}
Druha skupina metod jsou metody nejvyssiho zbytku.
Tyto metody vyuzivaji takzvne kvoty.
Kvota je pocet hlasu potrebny k zisku jednoho mandatu, za dve kvoty ziska strana dva mandaty...
Kvoty jsou tedy vzdy prirozena cisla.
Kazda strana nejprve ziska pocet mandatu rovny poctu kvot ktere se vejdou do jejiho volebniho zisku.
Diky vyberu kvoty (jak bude dale vysvetleno) muzou byt rozdeleny budto vsechny mandaty, nebo mene nez ma byt.
Pokud je jich rozdeleno mene, ziskavaji zbyle mandaty po jednom strany s nejvetsim zbytkem po deleni poctu ziskanych hlasu kvotou.
Existuji i kvoty ktere nezarucuji ze mandaru nebude rozdeleno vice nez ma, ty pak odebiraji mandaty stranam s nejmensim zbytkem.
Tyto kvoty se ale pouzivaji zridka.
Rozlisovaci vlastnosti techto systemu je zpusob urceni kvoty.
Kvota se vzdy urcuje dle celkoveho souctu hlasu a poctu volenych zastupitelu.
Minimalni kvotu lze definovat snadno, nesmi dovolit zisk vice mandatu nez ma byt rozdeleno.
Tato kvota se naziva Droopova, a je definovana jako prvni prirozene cislo vetsi nez $\dfrac{H}{N+1}$ kde $H$ je soucet vsech hlasu a $N$ pocet rozdelovanych mandatu.
Tuto kvotu lze odvodit nasledovne: Celkove muze byt (bez dorozdelovani dle zbytku) rozdeleno maximalne tolik mandatu, kolikrat se kvota cela vejde do celkoveho poctu hlasu.
Pro kvotu $\dfrac{H}{N+1}$ se kvota vejde do celkoveho poctu hlasu presne $N+1$ krat, pokud se kvota zvysi, uz se do celkoveho poctu hlasu $N+1$ krat nevejde, vejde se tedy maximalne $N$, maximalne muze byt rozdeleno $N$ mandatu.

Druhou casto pouzivanou kvoto je kvota Hareova, nekdy take nazyvana prirozena kvota. Ta je rovna pomeru celkoveho poctu hlasu a poctu udelovanych mandatu.
Jak ukazal Gallagher, strany s mensim volebnim ziskem vetsinou ziskaji pri pouziti Hareovy kvoty vice mandatu nez pri uziti Droopovy. Take dokazal, ze MAximalni kvota zalezi na poctu kandidujicich stran.\autocite{GAL1}
%TODO The Alabama Paradox The Population Paradox The New States Paradox

\section{Rovnost}
Rovnost muze nastat pri rovnosti hlasu, nebo pri rovnosti pomeru pri rozdelovani dle metody nejvyssich prumeru.
V realnych volebnich systemech se vyuzivaji dva zpusoby reseni teto situace, los a prideleni strane s vetsim poctem hlasu. 
Prideleni strane s vetsim poctem hlasu je samozrejme mozne jen pro rovnosti pomeru, ne pri rovnosti hlasu, ale ma vyhodu v tom ze vysledek je deterministicky.

\section{Volby v krajich}
V mnoha statech se ale rozdelovani dle vyse popsanych metod neprobiha na celostatni urovni, ale oddelene v jednotlivych oblastech.
Smysl tohoto opatreni je, ze obcane maji sve \uv{mistni} zastupce.
Bezne to v praxi funguje tak, ze podle celkoveho poctu hlasu odevzdanych v kazde oblasti, se celkovy pocet kresel nejprve rozdeli mezi jednotlive oblasti.\footnote{Rozdeleni mandatu mezi oblasti je opet obdoba stejne ulohy deleni mandaty mezi staty nebo strany}
V kazde z techto oblasti je prideleny pocet mandatu rozdelovan oddelene dle vyse rozebranych pravidel.
Nektere staty rozdelovani neprovadi naprosto oddelene, ale castecne oddelene.
Prikladem muze byt nize vysvetleny system kterym se v CR volil parlament az do roku 2000.
Toto rozdelovani v oblastech sice dava volicum \uv{mistni} zastupce, ma ale zasadni nevyhodu.
Male oblasti vedou k naruseni pomerovosti systemu.
V extremnim pripade oblasti kde je volen vzdy jeden zastupce jde o vyse popsany vetsinovy system. 
Ale i vetsi oblasti pomerovost narusi, jak bylo ukazano u vysvetleni pomeroveho systemu.
%TODO Porovnani metod na volbach s mnoha mandaty > podobnost, s malo mandaty > znacne rozdily

\chapter{Volebni system voleb do Poslancke snemovny}
\section{Popis systemu}
Od roku voleb v roce 2002 se u nad s pouziva system voleb v zcela oddelenych krajich.
Mandaty jsou nejprve rozdeleny Hareovou kvotou mezi jednotlive kraje, k tomu jsou pouzity soucty platnych hlasu v kazdem kraji.
Pote jsou vyrazeny veskere hlasy pro strany ktere nesplnili podminky pro zisk mandatu, tedy nedosahly \uv{petiprocentni hranice}.
V kazdem kraji je pak zcela nezavisle provedeno rozdelovani mandatu dle hlasu odevzdanych v danem kraji podle d'Hondtovy metody.


\section{Pomerovost soucasneho systemu}
V teto casti budeme zkoumat vysledky voleb z roku 2006. 
Data jsou prevzata ze stranek Ceskeho statictickeho uradu.\autocite{CSU}
Tyto volby totiz ukazaly nektere zvastni vlastnosti ceskeho volebniho systemu.
Vysledky voleb je ukazan v tabulce 1.
\begin{table}[tbp]
\catcode`\-=12
\begin{tabular}{|c|c|c|c|c|c|c|c|}
\hline
\multirow{2}{*}{Strana}  & \multicolumn{2}{|c|}{Skutecne Vysledky} & \multirow{2}{*}{Teoreticke mandaty} & \multicolumn{4}{|c|}{Rozdeleni dle ruznych metod} \\ \cline{2-3} \cline{5-8}
& Hlasy & Mandaty & & d'Hondt & Adams & Hare & Droop \\  
\hline
ODS & 1892475 & 81 & 75.25 & 76 & 75 & 75 & 76 \\ 
\hline
ČSSD & 1728827 & 74 & 68.74 & 69 & 68 & 69 & 69 \\ 
\hline
SZ & 336487 & 6 & 13.38 & 13 & 14 & 14 & 13 \\ 
\hline
KSČM & 685328 & 26 & 27.25 & 27 & 27 & 27 & 27 \\ 
\hline
KDU-ČSL & 386706 & 13 & 15.38 & 15 & 16 & 15 & 15 \\ 
\hline
\end{tabular}
\caption{Celostatni pohled na vysledky}
\end{table}
Jak je nazorne vydet rozdeleni mandatu jak ho predvedl Cesky volebni system neodpovida poctu obdrzenych hlasu.
Tento rozdil je zpusoben vyse zminenym rozdelovanim mandatu v krajich.
Lide v kazdem kraji voli sve poslance \uv{oddelene} za \uv{svuj}kraj, tim ale dochazi k ovlivneni vysledku.
Pro prozkoumani vyse zminenych vysledku tedy volby dvakrat prepocitame s predpokladem, ze pomer volicu jednotlivych stran je vsude stejny.
Provedeme prepocet jednou tak, jako kdyby republiku tvorilo 8 kraju o 25 mandatech (volebne nejvetsi kraj Praha) a kdyby ho tvorilo 40 kraju o 5 mandatech (volebne nejmensi kraj Karlovarsky).
Vyuzijeme stejnou metodu jako vyuziva soucasny system.
Vysledek je videt v tabluce 2 a 3.
\begin{table}
\begin{center}
\begin{tabular}{|c|c|c|}
\hline
Strana & V kraji & V cele republice\\
\hline 
ODS & 2 & 80\\
\hline
ČSSD & 2 & 80\\
\hline
SZ & 0 & 0\\
\hline
KSČM & 1 & 40\\
\hline
KDU-ČSL & 0 & 0\\
\hline 
\end{tabular} 
\caption{Vysledky v krajich po 5 mandatech}
\end{center}
\end{table}
\begin{table}
\begin{center}
\begin{tabular}{|c|c|c|}
\hline
Strana & V kraji & V cele republice\\
\hline 
ODS & 10 & 80\\
\hline
ČSSD & 9 & 72\\
\hline
SZ & 1 & 8\\
\hline
KSČM & 3 & 24\\
\hline
KDU-ČSL & 2 & 16\\
\hline 
\end{tabular} 
\caption{Vysledky v krajich po 25 mandatech}
\end{center}
\end{table}
Zde je jasne videt jak deleni mandatu v krajich zasadne ovlivnuje vysledky voleb.
To je spatne obzvlaste pro male strany ktere pak maji problem i pres prekonani umele hranice pro vstup nejake mandaty ziskat.
\chapter{Mozne upravy a zaver}
\section{Celostatni rozdelovani}
Slovensko se vyse zminenych problemu zbavilo tak, ze jejich volby probihaji celostatne.
Jak bylo videt vyse, pri celostatnim rozdelovani mandatu lze dosahnout vysoke pomerovoti s prakticky libovolnou metopbou deleni mandatu.
Nevyhoda pro volice je, ze kazdy dostava kandidatni listinu o 150 kandidatech (i kdyz to je dano spise optimismem politickych stran nez podobou systemu) a nema \uv{sve} poslance.
I kdyz toto muzeme povazovat za reseni problemu, podivejme se, jestli by bylo mozne zachovat oblastni rozdelovani mandatu a pri tom vyresit zminene problemy.
\section{Jiny zpusob deleni mandatu v krajich}
d'Hondtova metoda neni povazovana za nejvice pomerovou.
SL je v tomto ohlodu hodnocena jako jedna z nejproporcnejsich metod.\autocite{BEN}
Moznym zlepsenim situace je tedy zavedeni teto metody misto metody d'Hondtovy.
Problemem je ze v krajich jako je Karlovarsky neni rozdeleni ktere by presne odpovidalo rozdeleni hlasu casto vubec mozne.
Strana s 10 procenty hlasu nemuze dostat nic blizsiho svemu zisku nez 0 nebo 20 procent kresel.
Pohou zmenou metody tedy problem resit nelze, i kdyz by ho mohla mirnit.
\section{NUTS 2 kraje}
Jakl bylo vyse popsane, cim mensi kraj tim tezsi je v nem dosahnout pomerneho rozdeleni.
Dalsi moznosti uporavy systemu je tedy zvetseni volebnich oblasti.
Vytvoreni oblasti ktere by slouzili jen pro volby by ale melo nekolik problemu.
Prvni z neich je ze se ztraci puvodni vyznam voleb v krajich, tedy to aby volice zastupovali \uv{jejich} mistni zastupci s kterymi se ztotoznuji.
Umele rozdeleni republiky na \uv{volebni kraje} muze take vest k politickym bojum o jejich hranice a pocet.\footnote{Toto je videt napriklad v USA kde se prekreslovani hranic volebnih obvodu stalo duvodem vleklychpolitickych i soudnich boju.}
Je proto praktictejsi aby se volebni kraje shodovali s jiz existujicimi hranicemi.
Nabizi se pouzit NUTS 2 regiony, jak je uzivaji Evropske i Ceske instituce vcetne Ceskeho statictickeho uradu.
Kazdy z techto regionu je tvoren jednim nebo vice kraji.
Je jich 8 a kazdy z nich ma vice nez 1,1 milionu obyvatel.
Ani toto ale neresi vsechny problemy.
Jak bylo ukazano vyse ani regiony velikosti Prahy (Praha je jednim z NUTS 2 regionu), nejsou zarukou pomerovosti.
Takto velike regiony by se ale pri spravne zvolene metode deleni mandatu mohly pomerovosti blizit, ale stale jsou jeste prilis male na to aby presne odpovidaly poctum hlasu.
Navic zde opet narazime na neplneni puvodniho umyslu, vytvoreni oblasti se kterymi volici citi spojenost.
\section{Rozdelovani mandatu stran do kraju}
Reseni ktere by autor teto prace chtel predlozit je rozdelit mandaty na celostatni urovni a ziskat tak rozlozeni sil v parlamentu ktere by presne odpovidalo pomerum hlasu ktere byly odevzdany ve volbach.
Mandaty kazde strany by se pak delily mezi kandidaty dane strany v jednotlivych krajich podle poctu hlasu pro danou stranu.
Tento system by zarucoval, ze hlasy stranam ktere se dostanou do parlamentu by v zadnem kraji nebyly irrelevantni, jak tomu ted je u stran s malym ziskem v malych krajich, a to ani kdyby nakonec strana v kraji zadne mandaty neziskala. 
Zachovala by se prislusnost zvolenych poslancu krajum, ale misto toho aby byla prilusnost krajum prvni a pomerovost rozdeleni mandatu mezi strany druha, udela z rozdeleni mezi strany dulezitejsi bod.
Tento postup ma ale take sve nevyhody.
Problem deleni maleho poctu mandatu mezi strany se promenil mezi deleni maleho poctu mandatu mezi kraje.
To uz ale neovlivnuje vysledek voleb tak jak ho zname, tedy rodeleni kresel stranam.

Zde je, jak by dopadli volby v roce 2006 dle navrhovane metody s tim, ze k deleni mandatu mezi strany bude vyuzito metody d'Hondtovy (soucasna metoda) a mezi kraje metody Saint-Lague.
\begin{table}
\begin{tabular}{|c|c|c|c|c|c|c|c|}
\hline
Kraj & Soucasny & Navrhovany & ODS & ČSSD & SZ & KSČM & KDU-ČSL \\ \hline
Hlavní město Praha & 25 & 24 & 13 & 6 & 2 & 2 & 1 \\ \hline
Středočeský kraj & 23 & 22 & 10 & 7 & 1 & 3 & 1 \\ \hline
Jihočeský kraj & 13 & 13 & 5 & 4 & 1 & 2 & 1 \\ \hline
Plzeňský kraj & 11 & 12 & 4 & 4 & 1 & 2 & 1 \\ \hline
Karlovarský kraj & 5 & 5 & 2 & 2 & 0 & 1 & 0 \\ \hline
Ústecký kraj & 14 & 13 & 5 & 5 & 1 & 2 & 0 \\ \hline
Liberecký kraj & 8 & 7 & 3 & 2 & 1 & 1 & 0 \\ \hline
Královéhradecký kraj & 11 & 11 & 4 & 4 & 1 & 1 & 1 \\ \hline
Pardubický kraj & 10 & 11 & 4 & 4 & 1 & 1 & 1 \\ \hline
Vysočina & 10 & 10 & 3 & 4 & 0 & 2 & 1 \\ \hline
Jihomoravský kraj & 23 & 23 & 8 & 8 & 1 & 3 & 3 \\ \hline
Olomoucký kraj & 12 & 13 & 4 & 5 & 1 & 2 & 1 \\ \hline
Zlínský kraj & 12 & 12 & 4 & 4 & 1 & 1 & 2 \\ \hline
Moravskoslezský
kraj & 23 & 24 & 7 & 10 & 1 & 4 & 2 \\ \hline
Celkem  & 200 & 200 & 76 & 69 & 13 & 27 & 15 \\ \hline
\end{tabular}
\caption{Porovnani nove metody a soucasneho systemu}
\end{table}

\section{Zaver}
Cilem teto prace bylo vytvorit popis soucasneho volebniho systemu do Poslanecke snemovny Parlamentu Ceske republiky vcetne teorie pomeroveho zastoupeni, analyzovat soucasny system a navrhnout jeho upravy. Kazdy volebni system je ale kompromisem, tedy i navrhovany system ma sve vady. Autor ale veri ze jeho aplikace by vytvorila ferovejsi volebni prostredi a pritom zachovala krajovou prislusnot zvolenych poslancu.
\section*{Reference}
\printbibliography[heading=none]
\end{document}
